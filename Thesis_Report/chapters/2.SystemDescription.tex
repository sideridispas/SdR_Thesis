Introduction text about the system in general..
mpla mpla mpla

\section{Project Goals}
In order to better describe the system and the final device's functionalities, we have to start with a clear definition of this thesis' objectives. The main target of this project is  the design and implementation of a monitoring device that can take electrical measurements from the output of a Photovoltaic (PV) module and also get a few temperature points across the panel.\par
More specific, this thesis' final completed device will have to fulfil the following goals:

\begin{itemize}
  \item Accurate measurement of both Voltage and Current, on the output of each PV module
  \item Temperature measurement with up to 20 sensing points, to distribute across the surfaces of the module
  \item One measurement point per second, containing all above mentioned data
  \item Minimum influence of the existing system
  \item Final data collection system based on wireless communication
  \item Portable and easy to set-up device
  \item Full weather resistant packaging
\end{itemize}

At this point, some of the above mentioned objectives may seem more abstract that they should or not specified enough. But the target of this part of the report is to make totally clear for the reader which are the project's goals and keep the requirements fairly simple. As the report progresses, we will fully analyse each one of the system objectives in the corresponding chapter, with more technical and strict way.

\section{System Description}
 One important aspect that we have to take into consideration when designing the monitoring device is the fact that it should be easily added to an existing and functional set-up of PV modules. Therefore, first step is defining how a default PV module system is structured and find the proper place for the new device. As shown in the figure below, the output of a PV module is usually connected to a DC-DC converter and after that is wired, of course depending on the specific case and area, to a larger energy grid.
 
 -FIGURE(Default PV panel connection to the DC-DC and grid)
 
 It is clear that the place to add the measuring device should be just after the PV module output and before the DC-DC converter. In that way, we can accurately get the electrical measurements of the panel and at the same time minimise the influence of the existing system and keep the majority of the set-up identical. An example of that kind of  installation of the measurement device in an existing PV module system is shown in the figure below.
 
 -FIGURE(the diagram with multiple PVs connected to the A-boxes and then to the DC-DCs and after)
 
 In the above diagram we can see that the output of each individual panel is fed to the measuring device and after that directly connected to the DC-DC converter, as before. Also, we can see that each measuring device is capable of wireless communication with an access point and consequently have Internet connection, in order to upload the measured data to the cloud.
 
 As we defined the existing system and the location of the measuring device, we can focus more on the way that this device will operate. There are two main objectives that this instrument should fulfill and if we try to simplify them as much as possible, we can conclude to the following functions:
 
\begin{itemize}
 \item Take the measurements from the PV module
 \item Store the measured data
\end{itemize}
 
It is clear that each of these two functions includes a lot more specific steps and processes, but this simple system division can help us understand the problem that the measuring device targets to solve. An abstract view of the two parts of the system is given in the figure below.

-FIGURE(Measurement and Data part diagram)

In the next sections, we will focus more in depth to each individual part of the measuring device, in order to describe the system in general. 

\subsection{Measurements}
More information and details about the measurements' part.
Which are the functionalities of that part of the system?

\subsection{Data and Control}
More information and details about the measurements' part.
Which are the functionalities of that part of the system?